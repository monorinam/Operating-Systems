\documentclass[english]{article}
\usepackage[T1]{fontenc}
\usepackage[latin9]{inputenc}
\usepackage{babel}
\usepackage{graphicx}
\usepackage{subcaption}
\usepackage[left=0.5in,right=0.5in,top=0.8in,bottom=0.5in,footskip=.25in]{geometry}
\usepackage{ocr}
\usepackage{color}
\usepackage{bm}
\usepackage{booktabs}
\usepackage{amsmath}
%\usepackage{mathabx}
\usepackage{amssymb}
\usepackage[usenames,dvipsnames]{xcolor}
\usepackage{tikz}
\usepackage{titling}
\usepackage{listings} 
\usepackage{forest}
\usepackage{caption}
\usepackage{longtable}
\usepackage{enumitem}
\usepackage{courier}
\lstset{% 
   language=Matlab, 
   basicstyle=\small\ttfamily, 
} 

\newcommand{\rou}[1]{\ocrfamily\small{\textbf{\textcolor{red}{#1}}}\normalfont\normalsize}
\newcommand{\var}[1]{\ocrfamily\small{\textbf{\textcolor{gray}{#1}}}\normalfont\normalsize}
\newcommand{\refline}[1]{(\protect\tikz{\protect\fill[white] (0,-0.3em) rectangle (1.8em,0.3em);\protect\draw[#1] (0,0) -- (1.8em,0);})}
\newsavebox\CBox
\def\textBF#1{\sbox\CBox{#1}\resizebox{\wd\CBox}{\ht\CBox}{\textbf{#1}}}
\title{MATH 240: Discrete Structures 1 Assignment \#2}
\author{Monorina Mukhopadhyay (ID: 260364335)}
\begin{document}
\pgfkeys{/forest,
    tria/.append style={ellipse, draw},
  }
\maketitle
\section*{Problem 1}
Assume $\sqrt{91}$ is a rational number. Then $\exists m, n $ such that $\sqrt(91) = \frac{m}{n} \implies \frac{m^2}{n^2} = 91$ Here, 
m and n have no common factors (they are coprime numbers) or gcd(m,n) = 1.
This can be re-written as $91 n^2 = m^2 \implies m^2 = 13 \times 7 n^2$. Since both 13 and 7 are prime numbers, this means that
91 must divide $m^2$ Since 91 is a prime number, this means $91 | m$ This means m can be written as $m = 91k$, where $\frac{m}{91} = k$
Plugging this back into the relationship between 91, m and n: $91^2 k^2 = 91 n^2 \implies 91 k^2 = n^2 $. This means that $91|n$ as well, and
this contradicts the assumption that m and n are coprime. Therefore, $\sqrt{91}$ must be irrational.  
\section*{Problem 2}
\begin{enumerate} [label=\alph*]
\item Finding gcd(2016,208): Let b = 2016, a =208\\
$208 \times 9 + 144 = 2016$ \qquad (b=2016, a = 208) \\
$144 \times 1 + 64 = 208$ \qquad (b = 208, a = 144) \\
$ 64 \times 2 + 16 = 144$ \qquad (b = 144, a = 64) \\
$16*4 = 64$ \qquad (b=64, a = 16)\\
The gcd(2016,208) is 16
\item In class, it was shown that for a number to be expressed as $ma + nb$, divisibility by the gcd(m,n) was a necessary condition.
Since 1000 is not divisible by 16 the gcd(2016,208), there does not exist any such a and b to express 1000 as a linear combination of 2016 and 208.
\item 1024 is divisible by 16. So yes the numbers a and b exist such that $2016a + 208b = 1024$.\\
$1024 = 64*16$ and $16 = 144 - 64 \times 2 = 144 - 2 \times (208 - 144) = 3 \times 144 - 2 \times 208 = 3 \times (2016 - 9 \times 208) - 2 \times 208 = 3 \times 2016 - 29 \times 208$
So, from the two equations above we can get: $1024 = 64 \times 16 = 64 \times (3 \times 2016 - 29 \times 208) = 192 \times 2016 -  1856 \times 208$
\end{enumerate}
\section*{Problem 3} 
\begin{enumerate} [label=\alph*]
\item $32x \equiv 8$ mod 13. First, to check if there is an inverse: 32 and 13 are relatively prime, and so $32^-1$ exists.\\
Using Euclid's algorithm:\\
(b = 32, a = 13) \qquad $2 \times 13 + 6 = 32$ \\
(b = 13, a = 6) \qquad $2 \times 6 + 1 = 13$\\
$  1 = 13 - (6 \times 2) = 13 - 2(32 - 13 \times 2) = 5 \times 13 -2 \times 32 $
So, the inverse of 32 is -2 (mod 13) , and multiplying both sides by the inverse:
$ x \equiv -2 \cdot 8 = -16 \equiv 10$ mod 13. 10 mod 13 is just 10 and we can 
plug this back into the equation we get $32 \cdot 10 - 8 = 312 = 24 \cdot 13$ so x = 10 satisfies the congruency.
\item 169 and 39 have a common divisor. So, using definition of modulus operator:
$ 39 x + 169t = 65 \implies 3 x + 13 t = 5 \implies 3x \equiv 5$ mod 13. 3 and 13 are relatively prime, and so using Euclid's Algorithm:\\
(b = 13, a =3) \qquad $ 4 \times 3 + 1 = 13 \implies 1 = 13 - 4 \times 3$
So, the inverse of 3 is -4 (mod 13) and $-4 \equiv 9$ mod 13\\
$ x \equiv 5 \cdot -4 = -20 \equiv 6$ mod 13. In this case, 6 mod 13 is also 6 mod 169 since
the remainder in both cases is just 6. Plugging x = 6 into the original equation, it becomes $39 \times 6 - 65 = 234 - 65 = 169$, so x=6 works. 
\item $x^2 - 7x \equiv 10 $ mod 11 \\
$ \implies x^2 + 4x + 1 \equiv 0$ mod 11 since -7 mod 11 $\equiv$ 4 mod 11 and -10 mod 11 $\equiv$ 1 mod 11 \\
$\implies (x + 2)^2 - 3 \equiv 0$ mod 11 \\
$\implies (x + 2) ^2 \equiv 3 $ mod 11 \\
The smallest square that works to solve this is $x + 2 =5$ since 25 mod 11 $\equiv$ 3 mod 11.\\
So, $x \equiv 3$ mod 11. Similarly$ x + 2 = -5$ would also work, so $x \equiv -7 $ mod 11 $\equiv 4$ mod 11
\item 
\end{enumerate}
\section*{Problem 4}
The last digit of any number is that number mod 10. So need to find: $323^{4097} mod 10$ \\
323 mod 10 is 3, and this is the same remainder as 3 mod 10. So, $323^{4097} \equiv 3^{4097}$ mod 10. \\
4097 has two prime factors, 17 and 241. So $3^{4097}$ mod 10$ = 3^{17*241}$ mod 10$ = (3^{241})^{17} \equiv 3^{241} \equiv 3$ mod 10, and 3 is the last digit of $323^{4097}$
\section*{Problem 5}
Base case: n = 0. For n = 0, $ \frac{3^(n+1) - 1}{2} = 1$ and $3^n = 1$ So the base case holds\\
Induction case: Assume for k-th case, $1 + 3 + 3^2 ....+ 3^k = \frac{3^{k+1} - 1}{2} $ is true\\
For k + 1, $ 1 + 3 + 3^2 + ...+3^k + 3^{k+1} = \frac{3^{k+1} - 1}{2} + 3^{k+1} = \frac{3^{n+1} - 1 + 2.3^{k+1}}{2} = \frac{3^{n+2} - 1}{2}$
\section*{Problem 6}
\begin{enumerate} [label=\alph*]
\item A positive integer can be written as $N = a_k 10^k + a_{k-1} 10 ^ {k-1} + .... + 10 a_1 + a_0$ Also, $10 equiv 1 $ (mod 9) $\implies 10 \cdot 10 \equiv 1 \cdot 1 $ mod 9 (by the multiplication thm. of modular arithmetic, and by extension, $10^k \equiv 1^k = 1 $ mod 9\\
Therefore, $a_i 10^i \equiv a_i $ mod 9 , for $i = [1,k]$ and adding up all the $a_i 10^i$ terms gives N. $\Rightarrow N = a_k 10^k + a_{k-1} 10 ^ {k-1} + .... + 10 a_1 + a_0  \equiv a_k + a_{k-1} + a_{k-2} + .... + a_0 $ mod 9
\item From part a), it can be seen that for any multiple of 9, the sum of the digits will be congruent to the number itself mod 9, and so the sum of the digits must also be a multiple of 9.\\
Base case: n = 9. The sum of digits of 9 is 9 so the base case holds\\
Induction hypothesis: Assume that $n_1$ is a multiple of 9, and that the sum of its digits are some other multiple of 9 $n_2$. Assume that the assertion holds for $n_2$. This is the induction hypothesis \\
Proof: Since the assertion holds for $n_2$ and the sum of digits of $n_1$ gives $n_2$ the assertion also holds for $n_1$. Since for every multiple of 9, the sum of digits reduce to another multiple of 9, the assertion holds for all multiples of 9. 
For the next number $n_2 = n_1 + 9$, $n_2 \equiv 0$ mod 9, and $\sum_{i=0} ^{k}$
We need to prove that for $n = n_1 + 9$ the same holds true. 

\end{enumerate}

\section*{Problem 7}
Base case:  n = 3 is the smallest number of people for which the problem holds.
This is the base case.  Since all the distances are distinct, there must be one pair of two people who are closer to each other than the third person is to either of them. So, these two people throw their cakes at each other, and the third person avoids having cake thrown at them. \\
Induction hypothesis: Assume there are k people in the room, where k is odd, and that one person does not get a cake thrown at them. Let this person be 1, since it does not matter who it is.\\
Now, if there are k+2 people in the room, persons 2 - k are already closer to each other than to person 1. Now, one of the options happens:\\
\begin{itemize}
\item The two new people are at a minimum distance to each other and throw cakes at each other. 1 is unscathed.
\item The two new people find partners at a minimum distance within the set of persons {2,3,4...k} and each match up with a person at a minimum distance from them, and the newly unpaired people are closer to each other than to 1 and throw their cakes at each other, and 1 escapes a caking.
\item The two new people find partners at a minimum distance within the set of persons {2,3,4...k} and each match up with a person at a minimum distance from them, of the two unpaired people (let them be 3 and 5, since it does not matter who they are), 3 is closer to 1 than to 5, and 1 and 3 cake each other, and the 5 is spared here. 
\item One of the two new people (let k+1) is closer to 1 than the other (let, k+2), and 1 and k + 1 cake each other, and k + 2 is saved. 
In all of these cases, one person is always spared from having a cake thrown at them. 
\end{itemize}
\section*{Problem 8}
$ 51x \equiv 34$ mod 646 $\implies 51x = 646t + 34$\\
Finding the difference between x and y for two x,y such that $51x \equiv 34 $ and $51y \equiv 34$ mod 646. \\
$\implies 51(x-y) \equiv 0$ mod 646
This means that $51(x-y) = 646t_2 \implies 3(x-y) = 38t_2$ for any $t_2$. Since 38 is not divisible by 3, (x-y) must be divisible by 38 for this to hold. So the solutions must all have a difference of 38 between them. This means that there are 646/38 = 17 solutions. \\
Now, we have to show that at least one solution exists, for the above to be true.  From 51x = 646t + 34 $\implies 51x - 34 = 646t$. To find a solution easily, for t=2, $x = 26$. Therefore, one solution exists and there must be 17 solutions total, all with a difference of 38 from the next. 
\end{document}