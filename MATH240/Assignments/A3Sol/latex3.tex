\documentclass[english]{article}
\usepackage[T1]{fontenc}
\usepackage[latin9]{inputenc}
\usepackage{babel}
\usepackage{graphicx}
\usepackage{subcaption}
\usepackage[left=0.5in,right=0.5in,top=0.8in,bottom=0.5in,footskip=.25in]{geometry}
\usepackage{ocr}
\usepackage{color}
\usepackage{bm}
\usepackage{booktabs}
\usepackage{amsmath}
%\usepackage{mathabx}
\usepackage{amssymb}
\usepackage[usenames,dvipsnames]{xcolor}
\usepackage{tikz}
\usepackage{titling}
\usepackage{listings} 
\usepackage{forest}
\usepackage{caption}
\usepackage{longtable}
\usepackage{enumitem}
\usepackage{courier}
\lstset{% 
   language=Matlab, 
   basicstyle=\small\ttfamily, 
} 
\newcommand{\ndiv}{\hspace{-4pt}\not|\hspace{2pt}}
\newcommand{\rou}[1]{\ocrfamily\small{\textbf{\textcolor{red}{#1}}}\normalfont\normalsize}
\newcommand{\var}[1]{\ocrfamily\small{\textbf{\textcolor{gray}{#1}}}\normalfont\normalsize}
\newcommand{\refline}[1]{(\protect\tikz{\protect\fill[white] (0,-0.3em) rectangle (1.8em,0.3em);\protect\draw[#1] (0,0) -- (1.8em,0);})}
\newsavebox\CBox
\def\textBF#1{\sbox\CBox{#1}\resizebox{\wd\CBox}{\ht\CBox}{\textbf{#1}}}
\title{MATH 240: Discrete Structures 1 Assignment \#2}
\author{Monorina Mukhopadhyay (ID: 260364335)}
\begin{document}
\pgfkeys{/forest,
    tria/.append style={ellipse, draw},
  }
\maketitle
\section*{Problem 1}
 $$ N \equiv 2 mod 3  \qquad \qquad (1)$$
 $$ N \equiv 1 mod 5  \qquad \qquad (2)$$
 $$ N \equiv 4 mod 7  \qquad \qquad (3)$$
 First, I use the Chinese Remainder Theorem to solve (1) and (2). 1 can be written as 
 $$ 1 = 3 m_1 + 5 m_2$$ 
 [b = 5, a = 3] \qquad 5 = 3x1 + 2\\
 {}[b = 3, a = 2] \qquad 3 = 2x1 + 1\\
 1 = 3 - 2 = 3 - (5 - 3) = 2 $\cdot$ 3 - 5\\
 From the Chinese Remainder Theorem, $x = 3 m_1 a_2 + 5 m_2 a_1 = 3 \cdot 2 \cdot 1 - 5 \cdot 2 = -4 $ solves (1) and (2) \\ Since -4 is less than $n_1 n_2$, and x mod y $\equiv $ x if x < y, $-4 \equiv -4 mod 15 \equiv 11 mod 15$
\\ 
Now, the two equations to solve are:
$$ N \equiv 11 mod 15 $$
$$ N \equiv 4 mod 7$$
Using the Chinese Remainder Theorem again:
[b = 15, a = 7] \qquad 15 = 7 $\cdot$ 2 + 1 \\
1 = 15 - 7 $\cdot$ 2 \\
N = 4(15) - 7(11)(2)  = -94 $\equiv$ -94 mod 105 $\equiv$ 11 mod 105 = 11 \\
Check:\\
11 mod 3 $\equiv$ 2 mod 3\\
11 mod 5 $\equiv$ 1 mod 5\\
11 mod 7 $\equiv$ 4 mod 7

\section*{Problem 2}
\begin{enumerate} [label=\alph*]
\item Need to find x,y such that 60 | xy, 60 $\ndiv$ x and 60 $\ndiv$ y \\ By inspection, for any x < 60 and y < 60, such that xy = 60k (where k is an integer), this will work. \\ For example, for x = 15, y =4: 15 mod 60 $\not \equiv$ 0 mod 60 and 4 mod 60 $\not \equiv$ 0 mod 60 but 15 $\cdot$ 4 mod 60 $\equiv$ 0 mod 60. \\
The same also holds true for x = 12, y =10, where xy mod 60 $\equiv$ 0 mod 60, but x mod 60 $\not \equiv$ 0 mod 60 and y mod 60 $\not \equiv$ 0 mod 60
\item x, y integers, p prime and xy mod p $\equiv 0 \implies$ xy = pt \\
Assume p does not divide x $\implies$ gcd(p,x) = 1 $\implies$ 1 = $m_1$p + $m_2$ x \\
$\implies$ y = $m_1$py + $m_2$xy $\implies$ y = $m_1$py + $m_2$pt = p($m_1$y + $m_2$t) $\implies$
p | y. \\Since x was picked randomly, the same holds true if x and y are reversed. Therefore, if a prime number divides the product of two integers, it must divide at least one of the two integers.
\end{enumerate}
\section*{Problem 3}
Given: m | N and n | N $\implies$ N = m$t_1$ and N = n$t_2$ \\ Since m and n are relatively prime,
gcd(m,n) = 1 $\implies$ N = $a_1 m N + a_2 n N = a_1 m n t_2 + a_2 n m t_1 = (a_1 t_2 + a_2 t_1) mn \implies$ mn | N
\section*{Problem 4}
a > 2 and n $\geq$ 1, a, n $\in \mathbb{Z}$ \\
Need to prove that: $a - 1 | a^n - 1$ \\
Proof by induction:\\
Base case: $a > 2, n = 1; a^n - 1 =a - 1 \implies (a-1)|(a-1)$ so base case holds\\
Induction hypothesis: Assume $a - 1|a^k - 1 \implies a^k -1 = (a - 1)t$ for an integer t \\
Induction Step: For k+1, $ a^{k+1} - 1 = a^{k+1} - a^k + a^k -1 = a^k (a - 1) + (a^k -1) = a^k (a-1) + (a-1)t = (a-1)(a^k +t ) \implies a - 1 | a^{k+1} - 1$
\section*{Problem 5}
$ x \in {0,1,2....,38}$ For x to satisfy $x^{39} - x \equiv 0 $ mod 39 means that 39 | $x^{39} - x$. \\ 
Since 39 = 13 $\cdot$ 3, and gcd(13,3) = 1, this means that 13 | $x^{39} - x$ and 3| $x^{39} - x$ \\
This is easily proved as below: \\
Let $x^{39} - x = N$. 39 |N $\implies$ N = 39k = 13$\cdot$3k = 13 (3k) and N=3(13k) $\implies$ 13 | N and 3 | N
Solving these two equations using the Chinese Remainder Theorem:\\
13 = 3(4) + 1 $\implies$ 1 = 13 - 3(4) $\implies$  x = $x^{39} \cdot 13 - 3 \cdot 4 \cdot x^{39} = x^{39}$
$x = x^{39}$ is only true for x= 0  or x = 1. For any higher x, $x^{39}$ is always > x. So there are only two values of x for which the equation holds true.
\section*{Problem 6}
Using the dynamic programming algorithm for fast algorithm shown in class:
$ 22^{362} = (22 ^ {181}) $\\$= (22^2) (22^{180})^2 $\\$= 22^2 (22^{90})^4 $\\$= 22^2 (22^{45})^8 $\\$= 22^2 22^8 (22^{44})^8 $\\$= 22^2 22^8 (22^{22})^{16} $\\$ = 22^2 22^8 (22^{11})^{32} $\\$ = 22^2 22^8 (22^{32}) (22^{10})^{32} $\\$ = 22^2 22^8 22^{32} (22^5)^{64} $\\$ = 22^2 22^8 22^{32} 22^{64} (22^4)^{64} $\\$ = 22^2 22^8 22^{32} 22^{64} (22^2)^{128} $\\$ = 22^2 22^8 22^{32} 22^{64} (22)^{256} $  
$ 22^{362}$ mod 12 = $22^2 22^8 22^{32} 22^{64} (22)^{256}$ mod 12. \\
$22^1$ mod 12  = 10 mod 12 = 10. \\
$22^2$ mod 12  = $(22^1 \cdot 22^1)$ mod 12 \\ = (22 mod 12) (22 mod 12) mod 12 $\equiv 10 \cdot 10$ mod 12 $\equiv 4 mod 12$
\\$22^4$ mod 12 $= (22^2 mod 12)(22^2 mod 12) mod 12$ \\ $ \equiv 16 mod 12 \equiv 4 mod 12$ 
\\$22^8 mod 12$ = $(22^4 mod 12)(22^4 mod 12) mod 12 \equiv 16 mod 12 \equiv 4 mod 12$
\\ $\implies$, all higher powers of 22 will also reduce to 4 mod 12 since they can always be expressed as (4 mod 12)(4 mod 12) mod 12.\\
$22^{362} mod 12 = 22^2 22^8 22^{32} 22^{64} (22)^{256}$ mod 12 $\equiv (4 \cdot 4 \cdot 4 \cdot 4 \cdot 4) mod 12 \equiv 4^5 mod 12 \equiv 4 \cdot 4^2 \cdot 4^2 mod 12 \equiv 4 \cdot 4 \cdot 4 mod 12 \equiv$ (4 mod 12)(4 mod 12) $\equiv$ 4 mod 12.

 
\end{document}